\documentclass[11pt]{article}
\usepackage[utf8]{inputenc}
\usepackage[T1]{fontenc}
\usepackage{minted}
\usepackage{graphicx}
\usepackage{hyperref}
\usepackage{CJKutf8}

\author{Student: Brian Cheung bc32427 \\ Professor: Mohit Tiwari \\ TA: Antonio Espinoza \\ Department of Electrical \& Computer Engineering \\ The University of Texas at Austin}
\date{\today}
\title{EE379K Enterprise Network Security Lab 2 Report}
\hypersetup{
 pdfauthor={Student: Brian Cheung bc32427 \\ Professor: Mohit Tiwari \\ TA: Antonio Espinoza \\ Department of Electrical \& Computer Engineering \\ The University of Texas at Austin},
 pdftitle={EE379K Enterprise Network Security Lab 2 Report},
 pdfkeywords={},
 pdfsubject={},
 pdfcreator={},
 pdflang={English}}

\begin{document}

\maketitle
\newpage
\section*{Part 1 - Vulnerable Web Apps}
\label{sec:part-1}
The task was to implement
\subsection*{1a - Set up a web-service in a container}

\subsection*{1b - Getting familiar with strace}
One of the methods to detect an exploit on the DVWA is to look for suspicious system calls using \verb|strace|.
Since DVWA was ran with Docker, a process called \verb|containerd| executes the system calls for the container.

\noindent To get the PID of \verb|containerd|, the following was run:
\begin{minted}{bash}
  $ ps -ef | grep containerd
  root      1155     1  0 09:36 ?        00:00:08 /usr/bin/containerd
\end{minted}

\noindent \verb|strace| can be attached to \verb|containerd| and all of its forked child processes by running the following command:
\begin{minted}{bash}
  $ sudo strace -p 1155 -o strace.txt -f
\end{minted}

\begin{minted}{bash}
  $ docker run --rm -it -p 80:80 vulnerables/web-dvwa
\end{minted}

\noindent Injected bash command:
\verb|; echo "malware" > /tmp/maliciousfile|

\begin{minted}{bash}
  28552 execve("/bin/sh", ["sh", "-c", "ping  -c 4 ; echo \"malware\" > /t"...], [/* 9 vars */] <unfinished ...>
  ...
  28552 open("/tmp/maliciousfile", O_WRONLY|O_CREAT|O_TRUNC, 0666) = 3
  28552 fcntl(1, F_DUPFD, 10)             = 10
  28552 close(1)                          = 0
  28552 fcntl(10, F_SETFD, FD_CLOEXEC)    = 0
  28552 dup2(3, 1)                        = 1
  28552 close(3)                          = 0
  28552 write(1, "malware\n", 8)          = 8
  28552 dup2(10, 1)                       = 1
  28552 close(10)                         = 0
  28552 exit_group(0)                     = ?
  28552 +++ exited with 0 +++
\end{minted}

\section*{Part 2 - SELinux}
\label{sec:part-1}
The task was to implement
\subsection*{2a - Set up}

\section*{Conclusion}
\label{sec:conclusion}
The lab took about 40 hours which was a bit longer than expected.
It was pretty interesting learning about socket connections in different languages and how GET requests are built using socket connections.
I think some parts of the lab were a little unclear and needed further clarification.
Overall, this lab served its purpose in providing a more hands-on experience that helped improve my understanding of networking.

\nocite{*}
\bibliography{bibliography}
\bibliographystyle{ieeetr}
\end{document}
